\thispagestyle{empty}
\setlength{\parindent}{0in}
\setlength{\parskip}{0em}

\textbf{POLITECHNIKA RZESZOWSKA} \hfill Rzeszów, 2025 \\
\textbf{im. Ignacego Łukasiewicza} \\
\textbf{WYDZIAŁ MATEMATYKI I FIZYKI STOSOWANEJ} \\

\begin{center}
	\textbf{STRESZCZENIE  PRACY  DYPLOMOWEJ }
\end{center}

\textbf{Tytuł:} Analiza i prognozowanie opłat transakcyjnych sieci Bitcoina \\
\textbf{Autor:} inż. Daniel Krzysik \\
\textbf{Promotor:} dr inż. Dawid Jaworski \\
\textbf{Słowa klucze:} Bitcoin, blockchain, opłaty transakcyjne\\

Celem niniejszej pracy była analiza i prognozowanie opłat transakcyjnych w sieci Bitcoin. Przeprowadzono wstępną eksplorację danych blockchaina, pozyskanych bezpośrednio z węzła Bitcoin Core, a następnie przygotowano dane do dalszego przetwarzania. Zastosowano podejście zarówno wielowymiarowe, jak i jednowymiarowe wykorzystując, klasyczne algorytmy regresyjne, modele drzew decyzyjnych oraz model szeregów czasowych SARIMA. Uzyskane wyniki mogą wspierać podejmowanie decyzji przez użytkowników Bitcoina oraz stanowić podstawę do dalszych badań nad modelowaniem zjawisk blockchain.


% Wstawienie pionowej przestrzeni do połowy strony
\vfill
$$\noindent\rule{\textwidth}{2pt}$$
\vfill

\textbf{RZESZOW UNIVERSITY OF TECHNOLOGY} \hfill Rzeszów, 2025 \\
\textbf{FACULTY OF MATHEMATICS AND APPLIED PHYSICS } \\

\begin{center}
	\textbf{DIPLOMA THESIS ABSTRACT}
\end{center}

\textbf{Title:} Analysis and Forecasting of Bitcoin Network Transaction Fees\\
\textbf{Author:} Eng. Daniel Krzysik \\
\textbf{Supervisor:}  Dr Eng. Dawid Jaworski \\
\textbf{Key words:} Bitcoin, blockchain, transaction fees\\

The aim of this thesis was to analyze and forecast transaction fees in the Bitcoin network. An initial exploration of blockchain data obtained directly from the Bitcoin Core node was conducted, followed by data preprocessing. Both multivariate and univariate approaches were applied, using classical regression algorithms, decision tree models, and the SARIMA time series model. The obtained results may support decision-making by Bitcoin users and constitute a basis for further research on modeling blockchain phenomena.